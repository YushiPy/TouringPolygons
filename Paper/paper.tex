
\documentclass[a4paper]{article}

% Language
\usepackage[english]{babel}

% Math and symbols
\usepackage{amsmath, amssymb, mathtools}

% Algorithms
\usepackage[linesnumbered,ruled,vlined]{algorithm2e}

\SetKwIF{If}{ElseIf}{Else}{if}{:}{else if}{else}{end}
\SetKwFor{For}{for}{:}{end}
\SetKwRepeat{Do}{do}{while}
\SetKwProg{Function}{def}{:}{}

% Graphics and figures
\usepackage{graphicx}
\usepackage{subcaption}
\usepackage{caption}
\usepackage{float}
\usepackage{adjustbox}
\usepackage{tikz}

% Text formatting and layout
\usepackage{indentfirst}
\usepackage{fancyhdr}
\usepackage{quoting}
\usepackage{xcolor} % no need for [dvipsnames] unless you need extended colors

% URLs and hyperlinks
\usepackage{hyperref}
\usepackage{url}

% Code listings (optional, if you actually use it)
\usepackage{listings}

% Framed boxes (optional, if you use them)
\usepackage{tcolorbox}
\usepackage{mdframed}

% Citations
\usepackage{amsrefs} % or biblatex if preferred
\usepackage{cite}

\usepackage{multirow} % Permite criar tabelas com uma célula ocupando várias linhas

\usepackage{setspace}     % Para definir espaçamento entre linhas. (\onehalfspacing, \singlespacing, \doublespacing)

\usepackage{breakcites}   % Para permitir quebra de linha no meio de citações.

\usepackage[top=2cm, bottom=2cm, left=2cm, right=2cm]{geometry}

\usepackage{array}

% Comando para marcar o texto para revisão.
\newcommand{\rev}[1]{\textcolor{red}{#1}}

% Permite escrever aspas normais "text" em vez de ``text''
\usepackage[autostyle]{csquotes}
\MakeOuterQuote{"}


% Page style setup
\pagestyle{fancy}
\lhead{\footnotesize {\sc }}
\chead{}
\rhead{}
\lfoot{}
\cfoot{}
\rfoot{\thepage}

\renewcommand{\headrulewidth}{0.5pt}
\renewcommand{\footrulewidth}{0.5pt}

\begin{document}

\title{Touring a Sequence of Polygons}
\maketitle

\begin{abstract}

	Given a sequence of \(k\) polygons in the plane, a start point \(s\), and a target point, \(t\), we seek a shortest path that starts at \(s\), visits in order each of the polygons, and ends at \(t\). If the polygons are disjoint and convex, we give an algorithm running in time \(O(kn \log(n/k))\), where \(n\) is the total number of vertices specifying the polygons. We also extend our results to a case in which the convex polygons are arbitrarily intersecting and the subpath between any two consecutive polygons is constrained to lie within a simply connected region; the algorithm uses \(O(n k^2 \log n)\) time. Our methods are simple and allow shortest path queries from \(s\) to a query point \(t\) to be answered in time \(O(k \log n + m)\), where \(m\) is the combinatorial path length. We show that for nonconvex polygons this “touring polygons” problem is NP-hard. 
	
	The touring polygons problem is a strict generalization of some classic problems in computational geometry, including the safari problem, the zoo-keeper problem, and the watchman route problem in a simple polygon. Our new results give an order of magnitude improvement in the running times of the safari problem and the watchman route problem: We solve the safari problem in \(O(n ^ 2 \log n)\) time and the watchman route problem (through a fixed point \(s\)) in time \(O(n ^ 3 \log n)\), compared with the previous time bounds of \(O(n ^ 3)\) and \(O(n ^ 4)\), respectively.

\end{abstract}

\vfill

\section{Introduction}

A number of problems in computational geometry—including the zookeeper problem, the safari problem, and the watchman route problem—involve finding a shortest path, within some constrained region, that visits a set of convex polygons in an order that is given, or can be deduced. When region constraints permit it, a locally shortest path may travel straight through a polygon; otherwise, it “reflects” off edges or vertices of the polygons. The three problems mentioned above have in common that a path is globally shortest if it is locally shortest—i.e., it “reflects” appropriately whenever it changes direction.

In this paper we give the first polynomial-time algorithm for the general touring polygons problem: to find a shortest path, between two specified points, that visits in order a sequence of k possibly intersecting convex polygons while lying in some constrained regions. We crucially use the property that local optimality of a path implies global optimality. One consequence is that we never need to compute distances; we only need to compute reflections of points with respect to lines. If the given polygons are not convex, this property fails, and we prove that the touring polygons problem
is NP-hard.

Many geometric shortest path problems, e.g., shortestpaths among obstacles, are solved using the general technique of computing a shortest path map with respect to a fixed starting point s, which is a subdivision of the plane into regions such that the shortest paths from \(s\) to all points \(t\) in one region are combinatorially equivalent. One cannot directly apply this technique to our problem since the number of combinatorially distinct shortest paths in the touring polygons problem can be exponential in the input size, as we show in Theorem 3. We use instead a new technique based on subdividing the plane only according to the combinatorial type of the last step of the path. We call this a last step shortest path map. Our algorithms are based on computing the last step shortest path maps iteratively, one for each of the \(k\) polygons that must be visited, in the order they are given. Given the sequence of such maps, a query for the shortest path from \(s\) to any query point \(t\) can be answered in time \(O(k \log(n / k))\): a query for \(t\) in the last shortest path map tells us whether the path to \(t\) goes through, or reflects from \(P_k\), thus determining a query point for the previous shortest path map.

The general touring polygons problem (TPP) is as follows. We are given a start point \(s = P_0\), a target point \(t = P_{k + 1}\), and a sequence of polygons, \(P_1, \dots , P_k\), possibly intersecting, each considered to be a closed region in the plane (i.e., each includes interior plus boundary). For each \(i = 0, \dots , k\), we are given a simply connected polygonal region \(F_i\), called a “fence”, with \(F_i\) containing \(P_i\) and \(P_{i+1}\). We say that a path \(\pi_i\) visits polygon \(P_i\) if \(\pi_i \cap P_i \ne \emptyset\). We say that \(\pi_i\) visits the sequence \(P_1, \dots , P_k\) if there exist points \(q_1 \in P_1, q_2 \in P_2, \dots, q_k \in P_k\) such that \(q_1, \dots, q_k\) appear in order along \(\pi_i\). We let \(p_0 = s\) and let \(p_i\) denote the first point of \(\pi_i\) (i.e., the point closest to \(s\) along the path) that lies within \(P_i\) and comes after \(p_{i - 1}\) along \(\pi_i\); such points exist if \(\pi_i\) visits the sequence \(P_1, \dots, P_k\). Our goal is to compute a shortest path that begins at \(s\), visits \(P_1, P_2, \dots, P_k\), in that order, and arrives at \(t\), with each subpath of \(\pi_i\) from \(p_{i - 1}\) to \(p_i\) lying within the fence \(F_{i - 1}\). In the unconstrained TPP, each of the fences \(F_i\) is the entire plane, \(\mathbb{R} ^ 2\).

An \(i\text{-path}\) is a path that starts at \(s\) and visits the sequence of polygons \(P_1, \dots, P_i\). A fenced \(i\text{-path}\) to a point \(q \in F_i\) is an \(i\text{-path}\) satisfying the relevant fence constraints, i.e., having the subpath from \(p_{j - 1}\) to \(p_j\) lying within \(F_{j - 1}\), for \(j = 1, \dots, i\) and the subpath from \(p_i\) to \(q\) in \(F_i\). Using this notation, the TPP asks for a shortest fenced \(k\text{-path}\) to \(t\). For a point \(x\), let \(\pi_i(x)\) denote a shortest fenced \(i\text{-path}\) that ends at \(x\). We let \(n\) denote the total number of vertices in the polygons \(P_1, \dots, P_k\) and the fences \(F_0, \dots, F_k\). 

The formulation of the TPP allows for both the cycle and the path versions of the problem: If the destination point \(t\) is the same as the starting point \(s\), we are computing a shortest cycle (“tour”) through \(s\) visiting the \(P_i\)'s. If the order in which the polygons \(P_i\) must be visited is not specified, then the touring polygons problem becomes the Traveling Salesperson Problem with Neighborhoods, which is NP-hard. See [20].

The touring polygons problem can be modeled as a special kind of 3-dimensional shortest path problem among polyhedral obstacles. Imagine \(k\) very large sheets of paper stacked up in parallel planes orthogonal to the \(z\)-axis. The \(i\)-th sheet has a hole, \(P_i\), cut out of it; each sheet of paper is a polyhedral obstacle. The point \(s\) is placed at a \(z\)-coordinate just below the first sheet and the point \(t\) at a \(z\)-coordinate just above the \(k\)th sheet. Projecting this configuration to the \((x, y)\) plane yields the original touring polygons input. A solution to the touring polygons problem is the projection of a shortest path in 3-dimensions that starts from \(s\), threads its way through the polygonal holes, and arrives at \(t\).

\subsection{Summary of Results}

\begin{itemize}

	\item We give an \(O(k n \log(n / k))\) time \((O(n)\) space) algorithm for the unconstrained TPP with disjoint convex polygons \(P_i\). No polynomial-time algorithm was previously known for this problem. For fixed \(s\), the data structure supports \(O(k \log(n / k))\)-time shortest path queries to \(t\).
	
	\item We give an \(O(n k ^ 2 \log n)\) time (\(O(nk)\) space) algorithm for the general TPP with possibly intersecting convex polygons \(P_i\) and arbitrary fences \(F_i\). Actually, we only require convexity of the part of \(P_i\)'s boundary from which fenced rays may bounce.

	\item We show that the full combinatorial shortest path map for the TPP has worst-case size \(\Theta((n - k) 2 ^ k)\). It can be computed in output-sensitive time and supports \(O(k + \log n)\)-time shortest path queries.
	
	\item We show that the unconstrained TPP is NP-hard in general, if the polygons \(P_i\) are allowed to be nonconvex. A fully polynomial time approximation scheme follows from the application of results for shortest paths among obstacles in \(\mathbb{R} ^ 3\).

	\item We give substantional improvements in the running time for two classical problems in computational geometry, namely (a) 

	\item We give substantial improvements in the running time for two classical problems in computational geometry, namely (a) the safari problem, for which our results imply an \(O(n ^ 2 \log n)\) algorithm (improving upon \(O(n ^ 3)\)); and (b) the (fixed-source) watchman route problem, for which our results imply an \(O(n ^ 3 \log n)\) algorithm (improving upon \(O(n ^ 4)\)). (Our results imply also an improved bound - \(O(n ^ 4 \log n)\) instead of \(O(n ^ 5)\) - for the “floating” watchman route problem [26].) One of the main significances of our new method is not just that it yields a substantially faster algorithm for the watchman route problem, but also that it avoids using dynamic programming and complicated path “adjustments” arguments (which were the source of the original erroneous claims of the problem's polynomial-time solvability).
	
	\item Our results apply also to give the first polynomial-time solution to the parts cutting problem (below).

\end{itemize}

\section{APPLICATIONS AND RELATED WORK}

\subsection{The Parts Cutting Problem}

Suppose we have a sheet of some material such as wood or sheet metal, and a collection \(P_1, \dots, P_k\) of disjoint polygons on the sheet that must be cut out with a laser or saw as efficiently as possible. We limit the options by requiring that the edges of each polygon \(P_i\) be cut in one continuous cycle, starting at some point \(p_i \in \partial P_i\). The cutter starts at point \(s\) and the objective is to determine the points \(P_i\) that minimize the length of the polygonal chain \((s, P_1, p_2, \dots, P_k)\), since the total length of the cutter path is the length of this chain plus the sum of the perimeters of the parts \(P_i\) (and this sum is a constant). We thus want a minimum length path that visits the boundaries of all the polygons. We suppose, furthermore, that we are given the order in which the polygons must be visited. (Certain applications specify the order in which parts must be cut; without a given order, the problem is obviously NP-hard.) This problem is known as the parts cutting problem [16]. The parts cutting problem is exactly the unconstrained TPP with disjoint \(P_i\)'s and \(F_i = \mathbb{R} ^ 2, \forall i\). Our algorithm solves the parts cutting problem in time \(O(k n \log(n / k))\); see Section 3.

\subsection{The Safari and Zookeeper Problems}

For the safari and zookeeper problems, we are given a simple polygon \(P\), and disjoint convex polygons (“cages”) \(P_1, \dots, P_k\) inside \(P\), each sharing exactly one edge with \(P\). In the zookeeper problem we seek a shortest tour inside \(P\) that visits each of the \(P_i\)'s but never enters any of them. In the safari problem we seek a shortest tour inside \(P\) that visits each of the \(P_i\)'s, and is allowed to enter them. In this paper we consider only the “fixed-source” versions of these problems in which a starting point, \(s\), for the tour is given. Although these problems seem different from the TPP in that no ordering of the \(P_i\)'s has been specified, it is easy to argue that a shortest tour must visit the \(P_i\)'s in the same order as they meet the boundary of \(P\) ([8, 21]). The zookeeper problem, introduced by Chin and Ntafos [8], has an \(O(n \log n)\) time algorithm [3] utilizing the full shortest path map. The safari problem was introduced by Ntafos[21], who claimed an \(O(n ^ 3)\) time algorithm, which was then improved to \(O(n ^ 2)\) [24]. However, Tan and Hirata [28] found an error in the earlier analysis and presented the current best algorithm with running time \(O(n ^ 3)\). Our algorithm solves a much more general problem, and improves the running time for the safari problem to \(O(kn \log n)\).

\subsection{The Watchman Route Problem}

In the watchman route problem (WRP) we are given a simple polygon \(P\), and the goal is to find a shortest tour inside \(P\) so that every point in \(P\) is seen from some point of the tour. We consider the “fixed-source” case of the WRP, in which a point \(s \in \partial P\) is specified as the starting point of the tour; the “floating” WRP, with no point \(s\) specified, requires time \(O(n)\) times that of the fixed WRP, as shown recently by Tan [26]. The WRP was introduced by Chin and Ntafos [9, 7], who claimed in [7] an \(O(n ^ 4)\) time algorithm for the fixed WRP in a simple polygon. However, years later, there was a flaw discovered in their algorithm, and several attempts were made to correct it (some of which were themselves flawed). The best current algorithm for the fixed WRP, based on a relatively complex dynamic programming algorithm, is due to Tan, Hirata, and Inagaki [25] and runs in time \(O(n ^ 4)\). Our results imply a new, simpler algorithm for the fixed WRP that runs in time \(O(n ^ 3 \log n)\). In order to see that the WRP is a special case of the TPP, we recall the notion (e.g., from [7]) of essential cuts and the essential pockets they define: A cut, \(c_i\), with respect to \(s\) is a chord defined by extending the edge \(e\) incident on a reflex vertex, \(v_i\), where \(e\) is chosen to be that edge incident on \(v\) whose extension creates a convex vertex at \(v\) in the piece containing \(s\). The portion of \(P\) on the side of \(c_i\) not containing \(s\) is the pocket, \(P_i\), induced by the cut \(c_i\). A pocket is essential if no other pocket is fully contained within it. A tour sees all of \(P\) if and only if it visits all essential pockets. Let \(P_1, \dots, P_k\) denote the sequence of essential pockets, clockwise around \(\partial P\) starting from \(s\). It is known that the shortest watchman tour visits this sequence of essential pockets in order. We then get an instance of the TPP. by defining each fence \(F_i\)
to be the bounding polygon \(P\).

\section{THE UNCONSTRAINED TPP FOR
DISJOINT CONVEX POLYGONS}

\subsection{Local Optimality Conditions}

Clearly we only need to consider \(k\)-paths that are polygonal chains. Such a path is locally shortest
if bends occur only at points on the boundaries of the \(P_i\)'s, and for each such bend point
\(b \in \partial P_i\), moving \(b\) slightly in either direction on the boundary of \(P_i\) while keeping other bends fixed makes the path longer. This implies that for a bend point \(b\) on the interior of an edge \(e\) of \(P_i\), the angle between the incoming segment \((\overline{ab})\) and \(e\) must be equal to the angle between the outgoing segment \((\overline{bc})\) and \(e\) — i.e., the path must reflect on \(e\), so that \(b \in \overline{ac'}\), where \(c'\) is the reflection of \(c\) with respect to the line through \(e\).

For a bend point at a vertex \(v = e_1 \cap e_2\) the condition of being locally shortest is that the outgoing path segment from \(v\) must leave \(v\) in the cone, \(\gamma\), bounded by the reflected rays, \(r'_1\) and \(r'_2\), corresponding to the two edges.

The following is a special case of Lemma 8.

\textbf{Lemma 1}. For any \(p \in \mathbb{R} ^ 2\) and any \(i \in {0, \dots, k}\), there is a unique locally shortest \(i\)-path, \(\pi_i(p)\), to \(p\). Thus, local optimality is equivalent to global optimality.

\textbf{Proof}. Let \(T_i\) be the first contact set of \(P_i\), i.e., the points where a shortest \((i-1)\)-path first enters \(P_i\) after visiting \(P_1, \dots, P_{i - 1}\). Since the \(P_j\)'s are disjoint, \(T_i \subset \partial P_i\). One can show that \(T_i\) is connected:	

\textbf{Lemma 2}. Each first contact set \(T_i\) is a (connected) chain on the boundary of \(P_i\).

\subsection{The Last Step Shortest Path Map}

Locally shortest \(i\)-paths may leave points of \(T_i\) only in certain directions, either continuing in a straight line, or properly reflecting. For \(p\) in \(T_i\) let \(r^s_i(p)\) be the set of rays of locally shortest \(i\)-paths going straight through \(p\), and let \(r^b_i(p)\) be the set of rays of locally shortest \(i\)-paths properly reflecting at \(p\). (The mnemonic is “s” for straight, “b” for bounce.) Lemma 1 implies that for \(p\) in an edge of \(T_i\), \(r^s_i(p)\) and \(r^b_i(p)\) each contain a single ray, and for \(p\) a vertex of \(T_i\), \(r^s_i(p)\) contains a single ray and \(r^b_i(p)\) consists of a cone.

Let \(r_i(p) = r^s_i (p) \cup r^b_i (p)\) and \(R_i = \cup_{p \in T_i} r_i(p)\). The set \(R_i\) is an infinite family of rays with the property that each point in the plane is reached by exactly one ray; we say that \(R_i\) is a starburst with source \(T_i\). Associated with \(R_i\) is a subdivision, \(S_i\), of the plane that groups together points reached by rays of \(R_i\) that leave from the same vertex of \(T_i\), or leave from the same side of the same edge of \(T_i\). We call the polygonal subdivision \(S_i\) the last step shortest path map (with respect to \(s\)) associated with \(P_i\), since it encodes the information about the last segment in a shortest \(i\)-path from \(s\) to any point \(p\). In more detail, \(S_i\) decomposes the plane into cells \(\sigma\) of two types: (1) cones with an apex at a vertex \(v\) of \(T_i\), whose bounding rays are those of \(r^b_i (v)\); and (2) unbounded three-sided regions associated with an edge \(e\) of \(T_i\). Cells of type (2) can be further classified as being reflection cells or pass-through cells, depending on whether the rays reflect off \(e\) or enter the interior of \(P_i\) at \(e\). For cells of type (1), \(v\) is the source of the cell; for cells of type (2), \(e\) is the source. The union of all pass-through cells defines the pass-through region. Refer to Figure 1 for an example showing \(S_1\) and \(S_2\).

Using \(\{S_1, \dots, S_i\}\) we can readily compute a shortest \(i\)-path to any query point \(q \in \mathbb{R} ^ 2\) as follows: We locate \(q\) in \(S_i\). If the source of the cell \(\sigma\) containing \(q\) is a vertex, \(v\), of \(T_i\), then the last segment of \(\pi_i(q)\) is \(\overline{vq}\), and we recursively compute the shortest (\(i - 1\))-path to \(v\) (locating \(v\) in \(S_{i-1}\),etc). If the source of \(\sigma\) is an edge \(e\) of \(T_i\), then there are two subcases: (a) cell \(\sigma\) is a pass-through cell, in which case \(\pi_i(q) = \pi_{i-1}(q)\) and we recursively compute the shortest (\(i - 1\))-path to \(q\); or (b) cell \(\sigma\) is a reflection cell, in which case we let \(q'\) be the reflection of \(q\) in the line through \(e\), and recursively compute the shortest (\(i - 1\))-path to \(q'\); replacing the portion of this path from \(e\) to \(q'\) by the segment from \(e\) to \(q\) yields the shortest \(i\)-path to \(q\).

\textbf{Lemma 3}. Given \(S_1, \dots, S_i\), the path \(\pi_i(q)\) can be determined in time \(O(k \log(n/k))\) for any \(q \in \mathbb{R} ^ 2\).

\textbf{Proof}. The complexity of each \(S_i\) is clearly \(O(|P_i|)\) (for \(i = 1, \dots, k\)), and can be stored in a point location data stru-ture supporting \(O(\log |P_i|)\)-time queries for \(q\) in \(S_i\). The time to find \(\pi_i(q)\) is then \(O(\sum_{1}^{k} log(|P_i|))\), which attains its maximum when each \(|P_i| = n / k\).

We construct each of the subdivisions \(S_1, S_2, \dots, S_k\) iteratively, using the data structures \(\{S_1, \dots, S_{i-1}\}\) in the construction of \(S_i\). We store the subdivisions in a point location data structure.

The algorithm is very simple: For each vertex \(v\) of \(P_i\), we compute \(\pi_{i - 1}(v)\). If this path arrives at \(v\) from the inside of \(P_i\) then \(v\) is not a vertex of \(T_i\). Otherwise it is, and the last segment of \(\pi_{i - 1}(v)\) determines the rays \(r^b_i(v)\) and \(r^s_i(v)\) that define the subdivision \(S_i\). Thus, combining with Lemma 3 we have 

\textbf{Theorem 1}. For the unconstrained TPP for disjoint convex polygons with input size \(n\), a data structure of size \(O(n)\) can be built in time \(O(kn \log(n/k))\) that enables shortest \(i\)-path queries to any query point \(q\) to be answered in time \(O(i \log(n/k))\).

\section{THE GENERAL TPP ALGORITHM}

We turn our attention now to the general touring polygons problem with intersecting polygons and fence constraints. As stated in the introduction, we assume that fence \(F_i\) contains \(P_i\) and \(P_{i + 1}\), and we require that the subpath from \(p_i\) to \(p_{i+1}\) must lie in \(F_i\). We solve a more general case than that of convex \(P_i\)'s. Define the facade of \(P_{i + 1}\) in \(F_i\) to be \(\partial P_{i + 1} \cap c l (F_i - P_{i + 1})\). We assume that the facade of \(P_{i + 1}\) in \(F_i\) is a (single) convex chain; i.e., adding the segment to close the chain yields a convex polygon (possibly degenerating to a line segment). The facade represents the points on the boundary of \(P_{i + 1}\) from which rays in \(F_i\) can bounce.

\subsection{Combinatorial Facts}

In the presence of fences, a locally shortest path may bend at a fence vertex, with the usual condition for shortest paths inside a polygon — namely that the angle interior to the polygon be reflex. Intersecting polygons also complicate things: a path is locally shortest at an intersection point \(p\) of \(\partial P_{i-1}\) and \(\partial P_i\) if it is shorter than paths that visit \(P_{i-1}\) and then \(P_i\) close to \(p\).

We again use last step shortest path maps. As before, we let \(T_i\) be the first contact set of \(P_i\), i.e., the locus of points where a locally shortest fenced (\(i - 1\))-path first enters \(P_i\) after visiting \(P_1, \dots, P_{i-1}\), and then travelling though \(F_{i-1}\) to \(P_i\). Because polygons may intersect, first contact may occur when a locally shortest fenced (\(i - 1\))-path ends at a point that is already inside \(P_i\); thus \(T_i\) need not be part of \(P_i\)'s boundary. For example, if \(s\) is contained in \(P_1\), then \(T_1\) is just the point \(s\). In general, \(T_i\) will consist of the portions of \(T_{i-1}\) that lie inside \(P_i\) together with parts of the boundary of \(P_i\). As before, we define \(R_i\) to be the set of rays leaving points of \(T_i\) to begin locally shortest fenced \(i\)-paths. 

In order to prove that a locally shortest path is globally shortest, we need to prove some results on \(T_i\) and \(R_i\). We begin by describing \(T_i\) in terms of \(T_{i-1}\) and the fence \(F_{i-1}\). \(T_i\) can be described as the set of points \(p \in P_i\) s.t. there is a shortest path inside \(F_{i - 1}\) starting along a ray of \(R_{i - 1}\) from some point of \(T_{i - 1}\), and intersecting \(P_i\) for the first time at \(p\). Let \(A_{i - 1}\) be the last rays of such paths. We include in \(A_{i - 1}\) the rays of \(R_{i - 1}\) leaving points \(p \in T_{i - 1} \cap P_i\). Sources for the rays of \(A_{i - 1}\) are vertices of \(T_{i - 1}\), points on edges of \(T_{i - 1}\), and vertices of \(F_{i - 1}\). In the absence of fences, \(A_{i - 1}\) was a subset of \(R_{i - 1}\), but now \(F_{i - 1}\) intervenes. 

We will assume by induction that \(T_{i - 1}\) is a tree and \(R_{i - 1}\) is a starburst. We will prove that any two distinct rays of \(A_{i - 1}\) do not intersect; that \(R_i\) is a starburst, that \(T_i\) is a tree.

We begin by generalizing the uniqueness of locally shortest \(s-t\) paths in a polygon to the case where the source is a starburst.

\textbf{Lemma 4}. Suppose \(F\) is a simple polygon, and we have rays emanating in a starburst from a connected source \(T\) inside \(F\). Then any point \(p\) interior to \(F\) is reached by a unique locally shortest path starting from a ray emanating from a point of \(T\). 

\textbf{Proof}. This is true if all points in \(F\) are reached directly by rays of the starburst, i.e., every point \(p \in F\) is reached by a ray \(r\) of the starburst emanating from some \(t \in T\), with the line segment \(\overline{tp}\) contained in \(F\). Otherwise there must be a ray \(r\) of the starburst that travels in \(F\) to a reflex vertex \(v\) of \(F\) and then cuts off a hidden pocket \(P\) in \(F\). Locally shortest paths from the starburst to points of \(P\) go through \(v\), and locally shortest paths inside \(P\) from \(v\) are unique (this is the single source case). We can apply induction on \(F - P\) to conclude that locally shortest paths from the starburst are unique in \(F - P\). Furthermore, since the extension of ray \(r\) is in both parts of this partition, and provides a locally shortest path, thus no locally shortest path from either part may cross it.

\textbf{Lemma 5}. If \(R_{i - 1}\) is a starburst from a tree \(T_{i - 1}\), then no two distinct rays of \(A_{i - 1}\) intersect. 

\textbf{Proof}. This is more general than the lemma above since the intersection point of the rays may lie outside \(F_{i - 1}\). But the same proof idea applies. Consider a ray \(r\) of \(R_{i - 1}\) from point \(p \in T_{i - 1}\) that travels in \(F_{i - 1}\) to a reflex vertex \(v\) of \(F_{i - 1}\) and then cuts off a hidden pocket \(P\). If the portion of \(r\) up to \(v\) makes first contact with \(P_i\), then no points inside the pocket \(P\) are first contact points, and we are done by inducting on the smaller fence \(F_{i - 1} - P\). (Note that \(P_i\) still has a convex facade in this smaller fence.) On the other hand, suppose the portion of \(r\) up to \(v\) does not make first contact with \(P_i\). Consider the line segments \(l^+\) and \(l^-\) where \(l^+\) goes from \(v\) along \(r\) to the first contact with \(\partial F_{i - 1}\), and \(l^-\) goes from \(v\) backwards along \(r\), past \(p\), to the first contact with \(\partial F_{i - 1}\). (In other words, \(l^+\) forms the boundary of pocket \(P\).) We claim that because \(P_i\) has a convex facade it cannot intersect both \(l^+\) and \(l^-\). Thus we can apply induction on the smaller fence formed by cutting off the portion of \(F_{i - 1}\) beyond \(l^+\) or \(l^-\), whichever does not contain part of \(P_i\).

\textbf{Lemma 6}. If \(R_{i - 1}\) is a starburst from a tree \(T_{i - 1}\), then the rays of \(A_{i - 1}\) form an interval—i.e., the points at infinity reached by rays of \(A_{i - 1}\) form an interval.

\textbf{Corollary 1}. If \(R_{i - 1}\) is a starburst from a tree \(T_{i - 1}\) then \(T_i\) is connected. 

We begin with a discussion of what it means for rays to “bounce” at a vertex of \(T_i\). 

\textbf{Claim 1}. Let \(v\) be a vertex of \(T_i\), with edges \(e^1\) and \(e^2\) incident to \(v\) and consecutive in clockwise order around \(v\). Let \(R^1\) [\(R^2\)] be the set of rays leaving points on \(e^1\) [\(e^2\)], and entering the region between \(e^1\) and \(e^2\) in clockwise order. Let \(r^1\) and \(r^2\) be the limiting rays of these sets as the source approaches \(v\) along the edge. Then the rays that “bounce” (i.e., form locally shortest paths) at \(v\) consist of all rays between \(r^1\) and \(r^2\).

\textbf{Lemma 7}. If \(R_{i - 1}\) is a starburst from a tree \(T_{i - 1}\), then \(R_i\) is a starburst.

\textbf{Proof}. We already have that no two distinct rays of \(A_{i - 1}\) intersect. All of these become rays of \(R_i\), though their sources move from points on \(T_{i - 1}\) and \(F_i\) to points on \(T_i\). It remains to show that the various rays that are formed when rays of \(A_{i - 1}\) “bounce” off points of \(\partial P_i\) remain non-intersecting, and that they fill in the “gaps” to reach the whole plane. We address the issue of intersections first, dealing with rays that bounce from interior points of edges of \(T_i\), and then with rays that bounce from vertices of \(T_i\).

Two rays that bounce off edges of \(P_i\) cannot intersect each other because \(P_i\) has a convex facade. 

We next consider the case of one ray bouncing off an edge \(e\) of \(P_i\) intersecting one ray of \(A_{i - 1}\). Suppose that a ray \(r\) from a source point \(t_1\) of \(T_{i - 1}\) bounces at edge \(e\) of \(P_i\) to form ray \(b\). Suppose, for ease of description, that \(b\) bounces to the left of \(r\). Suppose by contradiction that \(b\) is intersected by a ray \(s\) coming from a point \(t_2\) of \(T_{i - 1}\) inside \(P_i\). This intersection must occur after \(s\) exits \(P_i\). Because \(s\) and \(r\) do not intersect, \(t_2\) must lie on the left of \(r\). Now t1 and t2 are connected by a path, say \(\gamma\), in \(T_{i - 1}\). The path \(\gamma\) must cross into \(P_i\). Let \(p\) be the first point of \(\gamma\) (traversed from \(t_1\) to \(t_2\)) on \(\partial P_i\). Because \(\gamma\) cannot cross \(r\) or \(s\), point \(p\) must lie between the points where \(r\) and \(s\) cross  \(\partial P_i\) . Let \(e'\) be the edge of \(T_{i - 1}\) containing \(p\). Consider the two rays \(u\) and \(v\) of \(R_{i - 1}\) that emanate from point \(p\) in \(R_{i - 1}\). We will argue that one of these rays must intersect \(r\) or \(s\), contradicting the fact that \(R_{i - 1}\) was a starburst. We crucially use the fact that \(u\) and \(v\) are reflections of each other in the edge \(e'\). Let \(r'\) and \(b'\) be rays emanating from \(p\) parallel to \(r\) and \(b\) respectively. The wedge in which \(u\) can live extends from \(r'\) counterclockwise to the part of \(e'\) inside \(P_i\) (which comes before \(e\)). The reflecting wedge in which \(v\) can live extends from \(e'\) to a ray that comes before \(b'\). Any possible ray \(v\) in this wedge will intersect either the path \(\gamma\) or the ray \(s\). This is a contradiction.

The only rays left to consider for possible intersections are those that “bounce” at vertices of \(T_i\) on \(\partial P_i\). These cannot produce an intersection, since, by Claim 1, they simply fill in the gaps between the rays emanating from interior points of edges. 

Finally, we must argue that the rays of \(R_i\) reach every point in the plane. This follows from Claim 1, together with the fact that \(T_i\) is connected (Corollary 1). 

\textbf{Corollary 2}. If \(R_{i - 1}\) is a starburst from a tree \(T_{i - 1}\), then \(T_i\) is a tree.

\textbf{Proof}. This follows from Corollary 1 plus the fact that the source of a starburst cannot contain a cycle (otherwise the ray of the starburst to a point inside the cycle would hit the cycle).

As an immediate consequence, we obtain

\textbf{Lemma 8}. For any \(i \in \{0, \dots, k\}\) and any \(p \in F_i\) there is a unique locally shortest fenced \(i\)-path to \(p\). Thus, local optimality is equivalent to global optimality.

\subsection{The Algorithm}

As before, we use a last step shortest path map. Actually, we use several such maps—with and without fences. We associate with the starburst \(R_i\) a subdivision of the plane, \(S^R_i\) , that groups together points reached by rays of \(R_i\) that leave from the same vertex of \(T_i\), or leave from the same side of the same edge of \(T_i\). A pass-through cell is one in which the rays leaving the source are inside \(P_i\) in a very small neighbourhood of the source. In particular, any cell of \(S^R_i\) whose source is a vertex or edge of some \(T_j\), \(j < i\), is a pass-through cell. For purposes of space efficiency, we group together all of the pass-through cells of \(S^R_i\) into one (possibly disconnected) “pass-through” region. Other cells of \(S^R_i\) have as their source either a vertex or an edge of \(T_i\).

The structure \(S^R_i\) tells us how shortest paths leave \(P_i\) ignoring the fence \(F_i\). To take the fence into consideration, we define a subdivision  \(S^F_i\) of \(F_i\) that groups together points reached by shortest paths in \(F_i\) that start from rays of \(R_i\), and whose last edges leave from the same vertex of \(F_i\), or leave from the same vertex of \(T_i\) or leave from the same side of the same edge of \(T_i\). Again, we group together all pass-through cells. (To add some intuition, for \(P_0 = s\), \(S^R_0\) has a single region containing all rays leaving \(s\), but \(S^F_0\) is a standard shortest path map inside the fence \(F_0\).)

Finally, we define the subdivision of the plane, \(S^A_i\), based on the rays \(A_i\) that arrive at \(P_{i + 1}\). Here we group together points of the plane reached by rays of \(A_i\) that emanate from the same vertex of \(T_i\), or the same vertex of \(F_i\), or from the same side of the same edge of \(T_i\). Again, we group together all pass-through cells.

We describe below how to compute these structures, but first we show how to use them to answer queries. 

\subsubsection{Answering Shortest Path Queries}

Given a point \(q \in F_i\), we can find a shortest fenced \(i\)-path to \(q\) by calling Query(q, \(S^F_i\)). We answer such a query by locating \(q\) in \(S^F_i\), and, based on the results, passing \(q\), or an appropriate reflection of \(q\), to appropriate level-(\(i - 1\)) data structures.

As we recurse, we will need to query \(S^A_{i-1}\) rather than \(S^F_{i-1}\), since our query point will not be guaranteed to lie inside \(F_{i - 1}\), but will be guaranteed to be hit by a ray of \(A_{i - 1}\). We say that a path respects \(F_i\) if it lies in \(F_i\) except that the last edge may possibly exit \(F_i\) after entering \(P_{i + 1}\). Our algorithm to construct the shortest path maps will also need to query \(S^R_i\). We thus describe a generic query in any of these structures. 

For \(X = F\) or \(A\) or \(R\), Query(\(q\), \(S^X_i\)) locates \(q\) in \(S^X_i\), and then follows these cases:

\begin{itemize}
	\item \textbf{Case 1}. \(q\) is in a cell whose root is a vertex \(v \in F_i\) or a vertex \(v \in T_i\). In this case, the last edge of the desired path is the line segment (\(v\), \(q\)), and we can find the actual path by using stored information at \(v\) (which we will have queried previously). 
	\item \textbf{Case 2}. \(q\) is in the “pass-through” region. In this case, a shortest fenced (\(i - 1\))-path to \(q\) automatically visits \(P_i\), and the portion of the path from the entry point of \(P_i\) to \(q\) respects \(F_i\). Thus, we desire a shortest fenced (\(i - 1\))-path to q. Note that \(q\) need not be inside \(F_{i - 1}\). However, it is reached by a ray of \(A_{i - 1}\), and so we obtain the correct answer by calling Query(\(q\), \(S^A_{i-1}\)).
	\item \textbf{Case 3}. \(q\) is in a cell whose root is an edge e of \(T_i\). Then the desired path to \(q\) bounces off edge \(e\). In this case, let \(q'\)be the reflection of \(q\) with respect to the line through edge \(e\). A shortest fenced (\(i - 1\))-path to \(q'\)automatically visits \(P_i\), and the final portion of it respects \(F_i\). Thus, we desire a shortest fenced (\(i - 1\))-path to \(q'\). The last segment of such a path will cross edge \(e\) at a point \(p\), say, and adding the line segment from \(p\) to \(q\) gives us the final path. Note that \(q'\)need not be inside \(F_{i - 1}\). However, it is reached by a ray of \(A_{i - 1}\), and so we obtain the correct answer by calling Query(\(q'\), \(S^A_{i-1}\)).
\end{itemize}




Provided we have the required planar subdivisions on hand, and their sizes are polynomial in \(n\), the above algorithm proves:

\textbf{Lemma 9}. For any point \(q\) in \(F_i\) we can find the last ray of a shortest fenced \(i\)-path to \(q\) in time \(O(i \log n)\). To compute the actual path we need to take into account the length of the path:

\textbf{Lemma 10}. For any point \(q\) in \(F_i\) we can find the shortest fenced \(i\)-path to \(q\) in time \(O(i \log n + m) = O(i \log n + f)\), where \(m\) is the combinatorial length of the path, and \(f = \sum_{j \le i} f_i\) is the total size of all the relevant fences. 

\subsubsection{Constructing the Last Step Shortest Path Maps}

Assuming we have \(S^F_{i-1}\) and \(S^F_j\), \(S^A)j\) for \(j < i\), we show how to construct \(S^A_{i-1}\) and \(T_i\) and \(S^R_i\), and then, from those, how to construct \(S^F_i\). 

Constructing \(S^A_{i-1}\). \(T_i\), \(S^R_i\). We compute all potential vertices of \(T_i\) that are not vertices of previous \(T_j\)'s as follows: throw in all vertices of the facade of \(P_i\) plus all intersection points of the facade of \(P_i\) and the facade of \(P_j\) for \(j < i\). For each such potential vertex \(v\), we find the last edge of a locally shortest fenced (i- 1)-path to \(v\) by calling Query(\(v\), \(S^F_{i-1}\)). On the basis of how the last edge of the path arrives at \(v\), we determine whether \(v\) is indeed a vertex of \(T_i\). If it is, the arriving ray becomes part of \(S^A_{i-1}\), and we compute the rays \(r^s_i(v)\) and \(r^b_i(v)\) to become part of \(S^R_i\). We preprocess \(S^A_{i-1}\) and \(S^R_i\) for point location queries.

Constructing \(S^F_i\). 

We preprocess \(F_i\) for ray shooting queries; this takes \(O(|F_i|)\) time, and allows queries in \(O(\log n)\) time [17]. We call Query(\(v\), \(S^R_i\)), for each reflex vertex \(v\) of \(F_i\), to find the last edge \(l\) of a shortest fenced \(i\)-path to \(v\) ignoring the fence \(F_i\). Using a ray shooting query in \(F_i\), we determine if \(l\) is contained in \(F_i\). If not, then we ignore \(v\). Otherwise we have found the shortest fenced \(i\)-path to \(v\). Let \(r\) be the continuation of \(l\) beyond \(v\) until its first intersection point with \(\partial F_i\). As in the proof of Lemma 4, we can identify the hidden pocket beyond \(r\). We can compute a standard shortest path map for source point \(v\) inside this pocket in linear time [17]. Putting together this information for all reflex vertices \(v\) of \(F_i\), and combining with the relevant portions of \(R_i\), yields the subdivision \(S^F_i\) of \(F_i\), which we then process for point location queries. 

Running time for the algorithm. Let \(f_i\) be the size of \(F_i\), and \(p_i\) be the size of the facade of \(P_i\) in \(F_i\). Let \(f = \sum_{i} f_i\), and \(p = \sum_{i} p_i\). In our general case \(f\) and \(p\) are \(O(n)\). Let \(t_i\) be the number of vertices of \(T_i\) that are not vertices of a previous \(T_j\), and let \(\sum_i t_i\).

\textbf{Lemma 11}. \(t\) is \(O(pk)\).

\textbf{Proof}. Each vertex of \(T_i\) that is not a vertex of some previous \(T_j\), \(j < i\), is an intersection point of an edge of the convex facade of \(P_i\) with an edge of the convex facade of \(P_j\) for some \(j < i\). Since the facades are convex, the number of intersection points is at most \(P_i\), \(\sum_{i,j} O(p_j + p_i)\), which attains its maximum value when \(p_i = \lceil {p/k} \rceil\) for each \(i\). Hence, the total number of these vertices is \(O(pk)\). 

The complexity of \(S^R_i\) is \(O(ti)\), the complexities of \(S^F_i\) and \(S^A_{i+1}\) are \(O(t_i + t_{i+1} + f_i)\). In the general case these are both \(O(nk)\) and, by Lemma 9 a query to find the last edge of a shortest fenced \(i\)-path (\(i \le k\)) to a point \(q\) takes \(O(k \log n)\) time.

We perform such a query for each vertex of each \(F_i\), and each vertex of each \(T_i\) that is not already a vertex of a previous \(T_j\). Thus we do \(p + f + t\) queries, and the total time required is \(O((p + f + t)k \log n)\). In our general case, this is \(O(n k^2 \log n)\).

\textbf{Theorem 2}. The TPP for arbitrary convex polygons \(P_i\) and fences \(F_i\) can be solved in time \(O(n k^2 \log n)\), using \(O(nk)\) space.

The Watchman Route Problem. For the watchman route problem in polygon \(P\), the polygons \(P_i\) are defined by the \(k = O(n)\) essential cuts of \(P\). Each \(P_i\) may have \(O(n)\) vertices, but \(p_i\), the size of the facade of \(P_i\), is just 2. Thus \(p\) is \(O(n)\). Each fence, \(F_i\), is all of \(P\), so \(f\) is \(O(n^2)\). Applying Lemma 11 we have that \(t\) is \(O(nk)\). From our analysis above, our algorithm takes time \(O((p + f + t) k \log n)\). Plugging in, we get \(O(n ^ 3 \log n)\). 

\textbf{Corollary 3}. The (fixed-source) watchman route in a simple polygon can be solved in time \(O(n ^ 3 \log n)\). The Safari Problem: For the safari problem, \(p\) is \(O(n)\).

Because the \(P_i\)'s are disjoint, \(t_i\) is just \(p_i\), and \(t\) is \(O(n)\) as well. Finally, fences can be defined such that \(f\) is \(O(n)\) ([3]). Plugging into the running time of \(O((p + f + t) k \log n)\) yields \(O(n k \log n)\). 

\textbf{Corollary 4}. The safari problem can be solved in time
\(O(nk \log n)\).

\end{document}
